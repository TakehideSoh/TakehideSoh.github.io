% Created 2013-10-13 日 14:22
\documentclass[a4j]{article}
                  
\providecommand{\alert}[1]{\textbf{#1}}

\title{プログラミング言語論および演習 (演習前半) @<br>** ライフゲーム (Game of Life) **}
\author{Takehide Soh}
\date{2013-10-13}
\hypersetup{
  pdfkeywords={},
  pdfsubject={},
  pdfcreator={Emacs Org-mode version 7.8.11}}

\begin{document}

\maketitle


\section*{ライフゲームとは?}
\label{sec-1}

\begin{itemize}
\item John Horton Conway が1970年に考案した二次元セルオートマトンの一種.
\item ライフゲームはいろいろな web page でデモをみることができる.
\begin{itemize}
\item \href{http://www.daiichi-g.co.jp/osusume/forfun/07_lifegame/07.html}{ライフゲーム (第一学習社)}
\item \href{http://math.shinshu-u.ac.jp/~hanaki/lifegame/}{Life Game (信州大学の花木先生のページ)}
\end{itemize}
\item 近年になっても新しい発見があり (すなわち未解決問題が多く存在する) 世界中の人がいろいろな事項を調査している.
\begin{itemize}
\item \href{http://pentadecathlon.com/lifeNews/index.php}{Game of Life News}
\begin{itemize}
\item 2012年には Marijn Heule 先生により6x6より小さい行列には \href{http://ja.wikipedia.org/wiki/%E3%82%A8%E3%83%87%E3%83%B3%E3%81%AE%E5%9C%92%E9%85%8D%E7%BD%AE}{エデンの園配置} が存在しないことが証明された.
\end{itemize}
\end{itemize}
\item \href{http://www.argentum.freeserve.co.uk/lex.htm}{Life Lexicon} には様々な初期状態が紹介されている!
\begin{itemize}
\item ``\href{http://www.argentum.freeserve.co.uk/lex_d.htm#diehard}{Diehard}'' は7個の \textbf{live} セルから始まる初期状態の中では長寿であり (最長寿?), 130世代で死滅する.
\end{itemize}
\item その他・参考資料
\begin{itemize}
\item \href{http://ja.wikipedia.org/wiki/%E3%83%A9%E3%82%A4%E3%83%95%E3%82%B2%E3%83%BC%E3%83%A0}{Wikipedia -- ライフゲーム}
\item \href{http://math.shinshu-u.ac.jp/~hanaki/lifegame/}{Wikipedia -- セル・オートマトン}
\end{itemize}
\end{itemize}
\section*{ライフゲームの設定・ルール}
\label{sec-2}

\begin{itemize}
\item ライフゲームでは \textbf{live} と \textbf{dead} の二つの状態を持つセル (cell) で構成される行列が与えられる.
\item セルは世代 (単位時間) の経過によって状態が変化する.
\item 各セルの次世代の状態は以下のルールによって決定される.
\begin{itemize}
\item 現在の状態が \textbf{dead} の場合
\begin{itemize}
\item 周囲のセルに \textbf{live} のセルがちょうど3つあれば次世代の状態は \textbf{live}
\item 他の状態の時は,次世代の状態も \textbf{dead}
\end{itemize}
\item 現在の状態が \textbf{live} の場合
\begin{itemize}
\item 周囲のセルに \textbf{live} のセルが2つか3つあれば次世代の状態は \textbf{live}
\item 他の状態の時は,次世代の状態は \textbf{dead}
\end{itemize}
\end{itemize}
\item なお行列の外の状態は \textbf{dead} であるものとする.
\item 例 ( \textbf{live} を 1 で表し,  \textbf{dead} を 0 で表す)
\end{itemize}


\begin{center}
\begin{tabular}{rrr}
\hline
 1  &  0  &  1  \\
 0  &  0  &  0  \\
 1  &  0  &  0  \\
\hline
\hline
 0  &  0  &  0  \\
 0  &  1  &  0  \\
 0  &  0  &  0  \\
\hline
\hline
 0  &  0  &  0  \\
 0  &  0  &  0  \\
 0  &  0  &  0  \\
\hline
\end{tabular}
\end{center}


  
\section*{Scala によるライフゲームの実装 (エンジン部分)}
\label{sec-3}
\begin{itemize}

\item 方針
\label{sec-3-1}%
\begin{itemize}
\item 簡単な機能から一つずつ実装する
\item 簡潔な実装を心がける
\begin{itemize}
\item 条件分岐がごちゃごちゃしたものを作らない!
\end{itemize}
\item 他の人がみても分かり易いコードを書く
\begin{itemize}
\item コメント文の追加
\item 分かり易い変数名
\end{itemize}
\end{itemize}

\item 実装のデザイン
\label{sec-3-2}%
\begin{itemize}
\item セルの状態 \textbf{live} を整数の1, \textbf{dead} を整数の0で表す.
\item 行列は Seq[Seq[Int]] で与えられるものとする.
\item \textbf{show} 関数 (表示,デバッグ用)
\begin{itemize}
\item (入力) Seq[Seq[Int]]型の行列 matrix
\item (出力) Unit. println によって行を一行ずつ表示する.
\item 出力例
\end{itemize}
\end{itemize}

\begin{verbatim}
scala> val a = Seq(0, 1, 0)      
scala> val b = Seq(1, 1, 0)      
scala> val c = Seq(0, 1, 0)      
scala> val matrix = Seq(a, b, c)                           
scala> show(matrix)              

// 出力結果
0 1 0
1 1 0
0 1 0
\end{verbatim}
\begin{itemize}
\item \textbf{getNeighborsState}
\begin{itemize}
\item (入力) セルの行番号 row: Int, セルの列番号 col: Int (実装によっては matrix: Seq[Seq[Int]] も)
\item (出力) Seq[Int]. 与えられたセルの周囲のセルの状態 (並びは任意)
\item 但し,行列の外の状態は一律に \textbf{dead} すなわち 0 であるものとする
\item 出力例
\end{itemize}
\end{itemize}

\begin{verbatim}
scala> val a = Seq(0, 1, 0)      
scala> val b = Seq(1, 1, 0)      
scala> val c = Seq(0, 1, 0)      
scala> val matrix = Seq(a, b, c)                           

scala> getNeighborsState(0, 0, matrix)
res0: Seq[Int] = Vector(0, 0, 0, 0, 1, 0, 1, 1)

scala> getNeighborsState(0, 1, matrix)
res1: Seq[Int] = Vector(0, 0, 0, 0, 0, 1, 1, 0)
\end{verbatim}
\begin{itemize}
\item \textbf{getNextCellState}
\begin{itemize}
\item (入力) 現在のセルの状態 current: Int, 周囲のセルの状態 neighbors: Seq[Int]
\item (出力) Int. 次世代のセルの状態
\item 出力例
\end{itemize}
\end{itemize}

\begin{verbatim}
scala> getNextCellState(1, Seq(0, 0, 0, 0, 0, 1, 1, 0))
res2: Int = 1
\end{verbatim}
\begin{itemize}
\item \textbf{trans}
\begin{itemize}
\item (入力) 現在の行列 matrix: Seq[Seq[Int]]
\item (出力) Seq[Seq[Int]] 次世代の行列
\item 出力例
\end{itemize}
\end{itemize}

\begin{verbatim}
scala> val a = Seq(0, 1, 0)      
scala> val b = Seq(1, 1, 0)      
scala> val c = Seq(0, 1, 0)      
scala> val matrix = Seq(a, b, c)                           

scala> trans(matrix)
res3: Seq[Seq[Int]] = Vector(Vector(1, 1, 0), Vector(1, 1, 1),
Vector(1, 1, 0))
\end{verbatim}

\begin{itemize}
\item 最終成果物の出力例
\end{itemize}

\begin{verbatim}
scala> val a = Seq(0, 1, 0)      
scala> val b = Seq(1, 1, 0)      
scala> val c = Seq(0, 1, 0)      
scala> val matrix = Seq(a, b, c)                           

scala> show(matrix)
0 1 0
1 1 0
0 1 0

scala> show(trans(matrix))
1 1 0
1 1 1
1 1 0

scala> show(trans(trans(matrix)))
1 0 1
0 0 1
1 0 1
\end{verbatim}
\end{itemize} % ends low level
\section*{回答例}
\label{sec-4}


\begin{verbatim}
 1:  // 与えられた行列を表示する
 2:  def show(matrix: Seq[Seq[Int]]) {
 3:    for (row <- 0 until matrix.size)
 4:      println(
 5:        (0 until matrix.size).map(col => if (matrix(row)(col) == 1) 1 else 0).mkString(" ")
 6:      )
 7:  }
 8:  
 9:  // matrix において行番号row, 列番号colにあるセルの周囲のセルの状態を返す
10:  def getNeighborsState(row: Int, col: Int, matrix: Seq[Seq[Int]]): Seq[Int] = {
11:    for (x <- -1 to 1; y <- -1 to 1 if (x != 0 || y != 0))
12:      yield (row + x, col + y) match {
13:      case (-1, _) => 0
14:      case (_, -1) => 0
15:      case (m, n) => if (m == matrix.size || n == matrix.size) 0 else matrix(m)(n)
16:    }
17:  }
18:  
19:  // 次世代のセルの状態を返す
20:  def getNextCellState(current: Int, neighbors: Seq[Int]): Int = {
21:    if (current == 1)
22:      if ((neighbors.count(_ == 1) == 2) || (neighbors.count(_ == 1) == 3)) 1 else 0
23:    else if (neighbors.count(_ == 1) == 3) 1 else 0
24:  }
25:  
26:  // 与えられた行列に対して次世代の行列を返す
27:  def trans(matrix: Seq[Seq[Int]]): Seq[Seq[Int]] = {
28:    assert(matrix.forall(_.size == matrix.size))
29:  
30:    for (row <- 0 until matrix.size)
31:      yield for (col <- 0 until matrix.size)
32:      yield getNextCellState(matrix(row)(col), getNeighborsState(row, col, matrix))
33:  }
\end{verbatim}
\section*{発展問題 (未実装・要検証)}
\label{sec-5}
\begin{itemize}

\item 4x4の行列について,与えた初期状態から動かない初期状態を列挙するプログラム
\label{sec-5-1}%

\item 4x4の行列について,振動する初期状態を列挙するプログラム
\label{sec-5-2}%

\item \href{http://en.wikipedia.org/wiki/Rule_30}{ルール30} の1次元のセルオートマトンをシミュレートするプログラム
\label{sec-5-3}%

\item 任意のルールを受け取る1次元のセルオートマトンシミュレーター
\label{sec-5-4}%
\end{itemize} % ends low level

\end{document}