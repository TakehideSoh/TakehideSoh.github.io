% Created 2015-11-29 日 20:09
\documentclass[11pt]{article}
\usepackage[utf8]{inputenc}
\usepackage[T1]{fontenc}
\usepackage{fixltx2e}
\usepackage{graphicx}
\usepackage{longtable}
\usepackage{float}
\usepackage{wrapfig}
\usepackage{rotating}
\usepackage[normalem]{ulem}
\usepackage{amsmath}
\usepackage{textcomp}
\usepackage{marvosym}
\usepackage{wasysym}
\usepackage{amssymb}
\usepackage{hyperref}
\tolerance=1000
\author{Takehide Soh}
\date{\today}
\title{Scarab: a SAT-based CP Systems}
\begin{document}

\maketitle
日本語ページは\href{./jp/index.html}{こちら}.

\section*{Overview}
\label{sec-1}
This page details \textbf{Scarab}, a prototyping tool for developing SAT-based systems. 
Features of Scarab are follows:

\begin{itemize}
\item \textbf{Expressiveness} Rich constraint modeling language.
\item \textbf{Efficiency} Optimized order encoding and native handling of BC/PB on Sat4j.
\item \textbf{Customizability} Its core part is written in around 1000 lines of Scala.
\item \textbf{Portability} Run on JVM.
\end{itemize}

\subsection*{How to use it}
\label{sec-1-1}
\begin{itemize}
\item Download \href{http://kix.istc.kobe-u.ac.jp/~soh/scarab/scarab-v1-6-9.zip}{the latest version}.
\end{itemize}
\url{./latin5-diagonal.gif}
\begin{itemize}
\item Write your own Scarab program. For instance, let's write a program solving Pandiagonal Latin Square \textbf{PLS(n)}.  
\begin{itemize}
\item a problem of placing different $n$ numbers into $n \times n$ matrix
\item such that each number is occurring exactly once
\item for each row, column, diagonally down right, and diagonally up right.
\end{itemize}
\end{itemize}
\begin{verbatim}
import jp.kobe_u.scarab.csp._
import jp.kobe_u.scarab.solver._
import jp.kobe_u.scarab.sapp._

var n: Int = 5
for (i <- 1 to n; j <- 1 to n)  int('x(i,j),1,n) 
for (i <- 1 to n) {
  add(alldiff((1 to n).map(j => 'x(i,j))))
  add(alldiff((1 to n).map(j => 'x(j,i))))
  add(alldiff((1 to n).map(j => 'x(j,(i+j-1)%n+1))))
  add(alldiff((1 to n).map(j => 'x(j,(i+(j-1)*(n-1))%n+1))))}

if (find)  println(solution.intMap)
\end{verbatim}
\begin{itemize}
\item Sava this program as \textbf{pls.sc}.
\item To run \textbf{pls.sc}, just execute it as follows, that's all !
\end{itemize}
\begin{verbatim}
scala -cp scarab-$VERSION.jar pls.sc
\end{verbatim}

\subsection*{More resources}
\label{sec-1-2}
\begin{itemize}
\item you can find more examples \href{./examples.html}{``example'' page}.
\item you can also find more documents \href{./doc.html}{``documents'' page}.
\end{itemize}

\subsection*{Note}
\label{sec-1-3}
\begin{itemize}
\item This software is distributed under the \href{http://opensource.org/licenses/bsd-license.php}{BSD License}. See \href{./LICENSE}{LICENSE} file.
\item scarab-<version>.jar includes \href{http://www.sat4j.org}{Sat4j} package and \href{http://bach.istc.kobe-u.ac.jp/sugar/}{Sugar} for the ease of use.
\begin{itemize}
\item We really appreciate the developers of Sat4j and Sugar!
\item Sat4j used for inference engine.
\item Sugar used for preprocessor (from 1.5.4).
\end{itemize}
\end{itemize}

\section*{Release Note}
\label{sec-2}

\begin{itemize}
\item\relax [2015.06.14] \href{./scarab-v1-6-9.zip}{ZIP of Scarab Package} -- Version 1.6.9 is released.
\label{sec-2-0-1}
\begin{itemize}
\item changes will be updated soon.
\end{itemize}

\item\relax [2015.05.25] \href{./scarab168.jar}{JAR of Scarab} -- Version 1.6.8 is released.
\label{sec-2-0-2}
\begin{itemize}
\item changes will be updated soon.
\end{itemize}
\item\relax [2015.02.08] \href{./scarab-v1-5-7.jar}{JAR of Scarab} -- Version 1.5.7 is released.
\label{sec-2-0-3}
\begin{itemize}
\item To run this version, Scala 2.11.* or higher is required.
\item Addition of new functions.
\begin{itemize}
\item UNSAT Core dectection in CSP level.
\item Nested commit.
\item Built-in optimization function.
\end{itemize}
\item Refactoring for some parts.
\end{itemize}

\item\relax [2015.01.09] \href{./scarab-v1-5-6.jar}{JAR of Scarab} -- Version 1.5.6 is released.
\label{sec-2-0-4}
\begin{itemize}
\item To run this version, Scala 2.11.* or higher is required.
\item Support non-contiguous domain.
\item Performance improvement.
\begin{itemize}
\item Order Encoding Module is tuned.
\item Native PB Constraint is tuned.
\end{itemize}
\end{itemize}

\item See history until 2014 \href{./history.html}{here}.
\label{sec-2-0-5}
\end{itemize}

\section*{Publications}
\label{sec-3}
\begin{itemize}
\item Scarab: A Rapid Prototyping Tool for SAT-based Constraint Programming Systems (Tool Paper)
\begin{itemize}
\item Takehide Soh, Naoyuki Tamura, and Mutsunori Banbara
\item In the Proceedings of the 16th International Conference on Theory and Applications of Satisfiability Testing (SAT 2013), LNCS 7962, pp. 429-436, 2013.
\end{itemize}
\item System Architecture and Implementation of a Prototyping Tool for SAT-based Constraint Programming Systems
\begin{itemize}
\item Takehide Soh, Naoyuki Tamura, Mutsunori Banbara, Daniel Le Berre, and Stéphanie Roussel
\item In the Proceedings of Pragmatics of SAT 2013 (PoS-13), 14 pages, July 2013.
\end{itemize}
\end{itemize}


\section*{Links for Related Tool}
\label{sec-4}

\begin{center}
\begin{tabular}{ll}
\href{http://www.sat4j.org}{Sat4j} & SAT solver in Java, which Scarab adopts!\\
\href{http://bach.istc.kobe-u.ac.jp/sugar/}{Sugar} & SAT-based CSP Solver using order encoding.\\
\href{http://bach.istc.kobe-u.ac.jp/copris/}{Copris} & Copris is a constraint programming DSL embedded in Scala.\\
 & It is also developed by our team!\\
\href{http://numberjack.ucc.ie}{Numberjack} & Constraint Programming System in Python\\
\href{http://lara.epfl.ch/web2010/scp}{SCP} & Constraint Programming in Scala using Z3\\
\href{http://code.google.com/p/scalasmt/}{scalasmt} & SMT in Scala using Z3\\
\href{https://bitbucket.org/oscarlib/oscar}{OscaR} & OR in Scala\\
\href{http://jacop.osolpro.com/}{JaCoP} & Constraint programming in Java and Scala\\
\href{http://www.emn.fr/x-info/choco-solver/}{Choco} & Constraint programming in Java\\
\href{http://jcp.org/en/jsr/detail?id\%3D331}{JSR 331} & Java Specification Requests: Constraint Programming API\\
\href{http://amit.metodi.me/research/bee/}{BEE} & a compiler which enables to encode finite domain constraint problems to CNF.\\
\href{http://jason.matf.bg.ac.rs/~mirkos/Mesat.html}{meSAT} & Multiple Encodings of CSP to SAT\\
\end{tabular}
\end{center}
% Emacs 24.3.1 (Org mode 8.2.10)
\end{document}
