% Created 2018-02-25 日 11:50
\documentclass[11pt]{article}
\usepackage[utf8]{inputenc}
\usepackage[T1]{fontenc}
\usepackage{fixltx2e}
\usepackage{graphicx}
\usepackage{grffile}
\usepackage{longtable}
\usepackage{wrapfig}
\usepackage{rotating}
\usepackage[normalem]{ulem}
\usepackage{amsmath}
\usepackage{textcomp}
\usepackage{amssymb}
\usepackage{capt-of}
\usepackage{hyperref}
\author{Takehide Soh}
\date{\today}
\title{Scala Lecture ** First Step  **}
\begin{document}

\maketitle
\setcounter{tocdepth}{1}
\tableofcontents


\section*{Overview of Scala}
\label{sec:orgheadline2}
\begin{itemize}
\item Scala is a statically typed programming language which realize
both object-oriented and functional programming.
\begin{itemize}
\item Scala is a pure object-oriented language in the sense that every value is an object.
\item Scala is also a functional language in the sense that every function is a value.
\end{itemize}

\item The creator of Scala is Martin Odersky (École Polytechnique
Fédérale de Lausanne (EPFL), Lausanne, Switzerland).
\begin{itemize}
\item He designed Scala in 2001.
\item On Feburuary 2018, the latest version of Scala is 2.12.
\item He said that:
\begin{itemize}
\item Scala was designed to show that a fusion of functional and
object-oriented programming is possible and practical. That’s
still its primary role. What has changed is that at the time it
came out FP was regarded as an academic niche. So it was
difficult to make the case that FP should be needed in a
mainstream language.
\end{itemize}
\end{itemize}

\item Who is using Scala?
\begin{itemize}
\item Many well known companies use Scala:
\begin{itemize}
\item LinkedIn
\item Twitter
\item Foursquare
\item Netflix
\item Tumblr
\item The Guardian
\item precog
\item Sony
\item AirBnB
\item Klout
\item Apple (added Sep., 2013)
\end{itemize}
\item This information is collected by the following pages.
\begin{itemize}
\item \url{http://www.scala-lang.org/old/node/1658}
\item \url{https://alvinalexander.com/scala/whos-using-scala-akka-play-framework}
\end{itemize}
\end{itemize}
\end{itemize}

\subsection*{Link}
\label{sec:orgheadline1}
\begin{itemize}
\item Official
\begin{itemize}
\item \url{https://www.scala-lang.org}
\end{itemize}
\item French
\begin{itemize}
\item \href{http://www.scala-lang.org/docu/files/ScalaByExample-fr_FR.pdf}{Scala par l'example}
\end{itemize}
\item English
\begin{itemize}
\item \href{http://www.artima.com/pins1ed/}{Programming in Scala, First Edition} (by Martin Odersky, Lex Spoon, and Bill Venners)
\item \href{http://programming-scala.labs.oreilly.com/index.html}{Programming Scala} (by Dean Wampler and Alex Payne)
\item \href{http://twitter.github.com/effectivescala/index-ja.html}{Effective Scala} (by Marius Eriksen, translated by Yuta Okamoto and Satoshi Kobayashi)
\item \href{http://twitter.github.com/scala_school/}{Twitter's Scala School!}
\item \href{https://wiki.scala-lang.org}{Scala Wiki}
\end{itemize}
\item News
\begin{itemize}
\item \href{https://www.lightbend.com/company/news/jaxenter-interview-with-scala-creator-martin-odersky-on-the-current-state-of-scala}{Interview with Scala creator Martin Odersky — The current state of Scaa}
\begin{itemize}
\item Documentation / Manuals / Translations に日本語のマニュアル等がある
\end{itemize}
\end{itemize}
\item Book
\begin{itemize}
\item (TODO)
\end{itemize}
\item Exercise
\begin{itemize}
\item \href{http://aperiodic.net/phil/scala/s-99/}{S-99: Ninety-Nine Scala Problems}
\item \href{http://blog.tmorris.net/scala-exercises-for-beginners/}{Scala exercises for beginners}
\end{itemize}
\end{itemize}

\section*{Install Scala into Your Computer}
\label{sec:orgheadline5}
\begin{itemize}
\item Even withtout installing, you can test some simple Scala program
via your browser
\begin{itemize}
\item \href{https://scastie.scala-lang.org}{scastie}
\item \href{https://scalafiddle.io}{scalafiddle}
\end{itemize}
\end{itemize}

\subsection*{Requirement}
\label{sec:orgheadline3}
\begin{itemize}
\item Java 8 is neccessary to be installed.
\item You can check it as follows (for Mac or Linux).

\item open terminal
\item type the followings
\end{itemize}
\begin{verbatim}
$ java -version

java version "1.8.0_121"
Java(TM) SE Runtime Environment (build 1.8.0_121-b13)
Java HotSpot(TM) 64-Bit Server VM (build 25.121-b13, mixed mode)
\end{verbatim}

\subsection*{Install Scala 2.12 (latest on Feburuary 2018)}
\label{sec:orgheadline4}
\begin{enumerate}
\item Open \url{https://www.scala-lang.org} in your browser.
\item Click ``Download''.
\item Scrool and find ``other ways to install Scala''. 
\begin{enumerate}
\item Click ``Download the Scala binaries for xxx'' --- xxx will be
your operation system.
\item Scala-2.12.4.tgz will be downloaded.
\end{enumerate}
\item Make a directory used in this lecture.
\begin{itemize}
\item (Note) Wherever you like. But if you do not have any
preference, please follow the instruction.
\end{itemize}
\begin{verbatim}
$ cd
$ mkdir scala18
$ cd scala18
\end{verbatim}
\item Move and un-archive ``Scala-2.12.4.tgz'' to your directory. 
\begin{verbatim}
$ mv <path downloaded>/Scala-2.12.4.tgz ./
$ tar xvzf Scala-2.12.4.tgz
$ ls
\end{verbatim}
\begin{itemize}
\item Can you find your Scala directory? Then it is okay!
\end{itemize}
\begin{verbatim}
$ ls
scala-2.12.4 scala-2.12.4.tar
\end{verbatim}
\item Set the PATH to Scala
\begin{verbatim}
echo 'export PATH=~/scala-2.12.4/scala-2.12.4/bin:$PATH' >>~/.profile
\end{verbatim}

\item Summary
\begin{verbatim}
$ cd ~
$ mkdir XXX
$ cd XXX
$ curl -O https://downloads.lightbend.com/scala/2.12.4/scala-2.12.4.tgz
$ tar xvzf scala-2.12.4.tgz
$ echo 'export PATH=~/XXX/scala-2.12.4/bin:$PATH' >>~/.profile
$ . ~/.profile
\end{verbatim}
\end{enumerate}

\section*{Verify if it is correctly working or not.}
\label{sec:orgheadline6}
\begin{itemize}
\item Type the followng.
\begin{verbatim}
$ scala
\end{verbatim}
\item Then you will see the following. Then, it is okay.
\begin{verbatim}
Welcome to Scala 2.12.4 (Java HotSpot(TM) 64-Bit Server VM, Java 1.8.0_121).
Type in expressions for evaluation. Or try :help.

scala>
\end{verbatim}
\end{itemize}


\section*{3 ways of running Scala}
\label{sec:orgheadline10}
\subsection*{REPL (Read Eval Print Loop)}
\label{sec:orgheadline7}
\begin{itemize}
\item (Overview)
\begin{itemize}
\item REPL is an interactive interface of Scala.
\end{itemize}
\item (For what?)
\begin{itemize}
\item simple program
\end{itemize}
\item (Usage)
\begin{itemize}
\item Type ``scala'' in terminal. Then, REPL will start and can be
used like the followings:
\end{itemize}
\begin{verbatim}
$ scala
Welcome to Scala 2.12.4 (Java HotSpot(TM) 64-Bit Server VM, Java 1.8.0_121).
Type in expressions for evaluation. Or try :help.

scala> 1+2
res0: Int = 3

scala> 3.toString
res1: String = 3

scala> 3.toDouble
res2: Double = 3.0

scala> :quit
\end{verbatim}
\end{itemize}

\subsection*{Run it like script}
\label{sec:orgheadline8}
\begin{itemize}
\item (Overview)
\begin{itemize}
\item scala provides script-like execution of its program without
(explicit) compilation.
\end{itemize}
\item (For what?)
\begin{itemize}
\item simple program
\end{itemize}
\item (Usage)
\begin{itemize}
\item Write a simple program in a text file. Then, we can run it as
follows.
\begin{itemize}
\item For instance, write the following and save it as ``helloWorld.sc''.
\begin{verbatim}
// (file) helloWorld.sc
object helloWorld {
  def main(args: Array[String]) {
    println("Hello World!")
  }
}
\end{verbatim}
\item Then, run it as follows.
\begin{verbatim}
$ scala helloWorld.sc 
Hello World!
\end{verbatim}
\item You will find a message ``Hello World!'' printed. Then, it is okay.
\end{itemize}
\end{itemize}
\end{itemize}

\subsection*{Compile and Run}
\label{sec:orgheadline9}
\begin{itemize}
\item (Overview)
\begin{itemize}
\item A standard way of running Scala program.
\end{itemize}
\item (For what?)
\begin{itemize}
\item simple to complex program
\end{itemize}
\item (Usage)
\begin{itemize}
\item Write a program in a text file. Then, we can run it as follows.
\begin{enumerate}
\item For instance, write the following and save it as ``helloWorld.scala''.
\begin{verbatim}
// (file) helloWorld.sc
object helloWorld {
  def main(args: Array[String]) {
    println("Hello World!")
  }
}
\end{verbatim}
\item Then, compile it as follows.
\begin{verbatim}
$ scalac helloWorld.scala
\end{verbatim}
\begin{itemize}
\item It will create the following 2 files.
\begin{verbatim}
helloWorld$.class
helloWorld.class
\end{verbatim}
\end{itemize}
\item Then, run it as follows (by spefifying the name of the
object containing the main method).
\begin{verbatim}
$ scala helloWorld
\end{verbatim}
\item You will find a message ``Hello World!'' printed. Then, it is okay.
\end{enumerate}
\end{itemize}
\end{itemize}
\end{document}
