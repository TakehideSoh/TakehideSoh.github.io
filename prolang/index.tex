% Created 2013-09-05 Thu 09:52
\documentclass[a4j]{article}
                  
\providecommand{\alert}[1]{\textbf{#1}}

\title{プログラミング言語論および演習 (演習前半)}
\author{Takehide Soh}
\date{2013-09-05}
\hypersetup{
  pdfkeywords={},
  pdfsubject={},
  pdfcreator={Emacs Org-mode version 7.8.11}}

\begin{document}

\maketitle



\section*{講義情報}
\label{sec-1}
\begin{itemize}

\item \textbf{講義名}: プログラミング言語論及び演習
\label{sec-1-1}%

\item \textbf{期間}: 平成25年度 後期 (10月4日〜11月15日)
\label{sec-1-2}%

\item \textbf{対象}: 情報知能工学科2回生
\label{sec-1-3}%


\item \textbf{教室}: 情報基盤センター 分館3F 演習室
\label{sec-1-4}%

\item \textbf{担当教員}: 宋剛秀
\label{sec-1-5}%

\item \textbf{スケジュール (全6回)}:
\label{sec-1-6}%
\begin{itemize}

\item (a) チーム:金曜4限 15:10 〜 16:40
\label{sec-1-6-1}%

\item (b) チーム:金曜5限 17:00 〜 18:30\\
\label{sec-1-6-2}%
\begin{center}
\begin{tabular}{lll}
\hline
 週     &  日付   &  内容                                                                                                                                                                                                  \\
\hline
 第1週  &  10/04  &  演習概要,\href{http://bach.istc.kobe-u.ac.jp/lect/ProLang/org/scala.html}{Scalaプログラミング入門}, \href{http://bach.istc.kobe-u.ac.jp/lect/ProLang/org/scala-list.html}{Scalaでリスト処理} (前半)  \\
\hline
 第2週  &  10/11  &  \href{http://bach.istc.kobe-u.ac.jp/lect/ProLang/org/scala-list.html}{Scalaでリスト処理} (後半)                                                                                                       \\
\hline
 第3週  &  10/18  &  \href{http://bach.istc.kobe-u.ac.jp/lect/ProLang/org/scala-recursive.html}{Scalaで再帰プログラミング} (前半)                                                                                          \\
\hline
 第4週  &  10/25  &  \href{http://bach.istc.kobe-u.ac.jp/lect/ProLang/org/scala-recursive.html}{Scalaで再帰プログラミング} (後半)                                                                                          \\
\hline
 第5週  &  11/01  &  \href{http://bach.istc.kobe-u.ac.jp/lect/ProLang/org/scala-complex.html}{Scalaで複素数計算}, 課題説明                                                                                                 \\
\hline
 第6週  &  11/15  &  Scala 演習,課題に関する質問など                                                                                                                                                                      \\
\hline
\end{tabular}
\end{center}




\end{itemize} % ends low level
\end{itemize} % ends low level
\section*{注意: 履修の前に必ず読んでください}
\label{sec-2}


\begin{center}
\begin{tabular}{l}
\hline
 \textbf{注意1}                                                                                                                                                                                                                            \\
\hline
 本演習は,情報基盤センター 分館(3F)で行います. 演習用の端末(iMac)を利用するには,情報基盤センター発行のアカウント (教務情報システムにログインする時につかったもの) が必要です.センター発行のアカウント通知書を必ずもってきてください.  \\
\hline
\end{tabular}
\end{center}


 

\begin{center}
\begin{tabular}{l}
\hline
 \textbf{注意2}                                                                                                                                                                                                                                                                                                                                              \\
\hline
 クラス (a) と (b) の分け方は以下の通りです.正しく履修してください.  (a) チーム:金曜4限 15:10 〜 16:40   学籍番号下1桁が奇数の学生 (入学年度は関係なし) (b) チーム:金曜5限 17:00 〜 18:30   学籍番号下1桁が偶数の学生 (入学年度は関係なし) ※ (a) と (b) を交換することはできません.再履修などの理由がある場合は申し出てください.         \\
\hline
\end{tabular}
\end{center}
\section*{最新情報}
\label{sec-3}
\begin{itemize}

\item 平成24年度のWebページを公開.(H24.10.04)
\label{sec-3-1}%

\item 課題問題とその提出方法 (H24.11.15)
\label{sec-3-2}%

\end{itemize} % ends low level
\section*{演習概要}
\label{sec-4}
\begin{itemize}

\item 代表的な論理型プログラミング言語であるPrologの演習を行います. Prologプログラミングの基礎を身につけることが目標です. Prologは,主に人工知能などの分野で使用されるプログラミング言語です. この演習ではPrologコンパイラ処理系として, SWI-Prologを使用します. SWI-Prologは,既に演習室のiMac (コマンド:/opt/local/bin/swipl)にインストール済です.
\label{sec-4-1}%

\end{itemize} % ends low level
\section*{演習情報}
\label{sec-5}
\begin{itemize}

\item WEBテキスト: Prolog入門
\label{sec-5-1}%

\item 成績評価: 「出席(10\%〜20\%)」 + 「課題(1つだけ) 」で総合的に評価
\label{sec-5-2}%

\item 履修用件: 予備知識は不要(でも,できれば数理論理学の基礎的な知識をもっていることが望ましい)
\label{sec-5-3}%

\item 参考資料:\\
\label{sec-5-4}%
「Prologの技芸」L.Sterling,E.Shapiro著 共立出版
  神戸大学Prologページ
 Prolog Cafe: PrologからJavaへのトランスレータ処理系



\end{itemize} % ends low level
\section*{使用するSWI-Prolog処理系}
\label{sec-6}
\begin{itemize}

\item SWI-Prolog ホームページ
\label{sec-6-1}%

\item SWI-Prologマニュアル
\label{sec-6-2}%

\item K-Prolog 組込み述語リスト(日本語)
\label{sec-6-3}%

\end{itemize} % ends low level
\section*{その他}
\label{sec-7}
\begin{itemize}

\item Emacsのshell中でPrologを起動
\label{sec-7-1}%

\item EmacsをDockから起動
\label{sec-7-2}%

\item ESC-x shellと入力しリターン\\
\label{sec-7-3}%
(ESC-xとは,escキーを押して,その後,xキーを押すという意味です)

\item するとshellが立ち上がるので,以下のようにswiplと入力\\
\label{sec-7-4}%
\begin{verbatim}
 1:  % /opt/local/bin/swipl
 2:  Welcome to SWI-Prolog
 3:  
 4:  For help, use ?- help(Topic). or ?- apropos(Word).
 5:  
 6:   ?- 
 7:  Prologの終了
 8:   ?- halt.
 9:  shellの終了
10:  % exit
\end{verbatim}

\end{itemize} % ends low level
\section*{Emacsの基本操作}
\label{sec-8}


\begin{center}
\begin{tabular}{ll}
\hline
 ファイルを開く      &  C-x C-f ファイル名 + リターン                           \\
 画面を上下に二分割  &  C-x 4 f + リターン                                      \\
 ファイルを保存      &  C-x C-s                                                 \\
 入力クリア          &  C-g                                                     \\
 次ぎの行            &  C-n                                                     \\
 前の行              &  C-p                                                     \\
 行の先頭            &  C-a                                                     \\
 行の末尾            &  C-e                                                     \\
 カット              &  C-k                                                     \\
 ペースト            &  C-y                                                     \\
 文字検索            &  C-s (C-g でキャンセル,リターンで決定,C-s で次の候補)  \\
 キーバインド表示    &  M-x describe-bindings                                   \\
\hline
\end{tabular}
\end{center}


注意:C-xとは,controlキーを押しながらxキーを押すという意味です.
\section*{EmacsでPrologプログラムに色をつけて見やすくするには}
\label{sec-9}

以下の1行を,.emacs.elファイルに追加してください. このファイルがない場合は,作成してホームディレクトリ直下に配置してくだ さい. この設定を行うと,Perl で書かれたファイル (拡張子 .pl) も prolog-mode で開いてしまうので注意してください.
(add-to-list `auto-mode-alist `(``\\.pl\$'' . prolog-mode))

\end{document}